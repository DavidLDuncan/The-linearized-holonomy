\documentclass{article}



\usepackage{amscd,amssymb,verbatim,graphicx,stmaryrd,amsthm,color} 
\usepackage{hyperref}


%\renewcommand{\normalsize}{\fontsize{13pt}{14.5pt}\selectfont} %Changes the font size to 13pt, I am not sure what the 14.5 pt slot does, but it appears to be desireable to have it here.



%XY pic stuff
\input xy    %Loads xy-Pic
\xyoption{curve}  %Inputs curve things (See section 8 of manual)
\xyoption{arc} %Fancy version of the previous
\xyoption{frame} %Frame extension. Needed for frm command
\xyoption{tips} %For fancy arrows
\xyoption{arrow} %Needed for \ar command 
\xyoption{color} %Color



%\usepackage{mathpazo} %Changes font to something nice.

\newtheorem{theorem}{Theorem}[section]  %This keeps track of the section in the naming of the theorems
%\newtheorem{theorem}{Theorem}
\newtheorem{lemma}[theorem]{Lemma}  %This creates a lemma environment and numbers it along with the theorems
\newtheorem{proposition}[theorem]{Proposition}
\newtheorem{corollary}[theorem]{Corollary}
\newtheorem*{conjecture}{Atiyah-Floer Conjecture} %This creates a theorem environment that does no label it along with the other theorems.
\newtheorem*{qconjecture}{Quilted Atiyah-Floer Conjecture} %This creates a theorem environment that does no label it along with the other theorems.
\newtheorem*{qcconjecture}{Quilted Atiyah-Floer Conjecture (Chain Level Version)} %This creates a theorem environment that does no label it along with the other theorems.
\newtheorem*{theorema}{Theorem A}
\newtheorem{theoremb}[theorem]{Main Theorem}
\newtheorem*{theoremc}{Theorem C}
%\newtheorem{exercise}{Exercise} %This invents an exercise environment that numbers independently of the theorems
\newtheorem{remark}[theorem]{Remark}
\newtheorem{example}[theorem]{Example}
%\newtheorem{remarkstar}[remark]{Remark$\left.\right.^*$}
\newtheorem*{remarksnolabel}{Remarks}
\newtheorem*{remarknolabel}{Remark}
\newtheorem{remarkslabel}{Remarks}











%\newcounter{theexample} \setcounter{theexample}{1}% This creates an example environment that numbers independently of anything else and that also allows me to play with the font (I don't know how to do this with \newtheorem
%\newenvironment{example}[1][]{ \begin{trivlist}
 %\hangindent=1.2cm  \item[ \hskip \labelsep \indent \indent {\bf  Example
%\arabic{theexample}. #1} \refstepcounter{theexample}]}{\end{trivlist}} 


%\newcounter{theremark} \setcounter{theremark}{1}%This creates a numbering for the remark environment I define below

%\newenvironment{remark}[1][Remark]{\begin{trivlist} %This creates a remark environment that is different from the example environment in that it does not italicize the contents. Otherwise they appear to be the same (though numbered differently.
%\item[\hskip \labelsep {\bfseries #1 \arabic{theremark}.}\refstepcounter{theremark}]}{\end{trivlist}} %Try to number these along with the theorems, but keep the text non-italicized. Perhaps I can call the theorem counter some-how. See the LaTeX manual. 





%\newenvironment{remark}[1][Remark.]{\begin{trivlist}
%\hangindent=1.2cm  \item[\hskip \labelsep \indent \indent { \bfseries #1}]}{\end{trivlist}}



%\newenvironment{example}[1][Example.]{\begin{trivlist}
%\hangindent=.68cm  \item[\hskip \labelsep   \indent { \bfseries #1}]}{\end{trivlist}}


\newcounter{theexercise} \setcounter{theexercise}{1}
\newenvironment{exercise}[1][]{ \begin{trivlist}
 \small\it \item[\hskip \labelsep  {\small \bfseries  Exercise
\arabic{theexercise}. #1} \refstepcounter{theexercise}]}{\end{trivlist}} 


%\newenvironment{exercise}[1][Exercise.]{\begin{trivlist}
 %\small\it \item[\hskip \labelsep   {\small \bfseries #1}]}{\end{trivlist}}



%\newenvironment{remark}[1][Remark.]{\begin{trivlist}
%\item[\hskip \labelsep {\bfseries #1}]}{\end{trivlist}}
%\newenvironment{remarkstar}[1][Remark.$\left.\right.^*$]{\begin{trivlist}
%\item[\hskip \labelsep {\bfseries #1}]}{\end{trivlist}}
\newenvironment{notation}[1][Notation.]{\begin{trivlist}
\item[\hskip \labelsep {\bfseries #1}]}{\end{trivlist}}
\newenvironment{notationa}[1][Notation and More Inner Products.]{\begin{trivlist}
\item[\hskip \labelsep {\bfseries #1}]}{\end{trivlist}}
\newenvironment{motivation}[1][Motivation.]{\begin{trivlist}
\item[\hskip \labelsep {\bfseries #1}]}{\end{trivlist}}
%\newenvironment{exercise}[1][Exercise.]{\begin{trivlist}
%\item[\hskip \labelsep {\bfseries #1}]}{\end{trivlist}}


%\newcommand{\qed}{\nobreak \ifvmode \relax \else
    %  \ifdim\lastskip<1.5em \hskip-\lastskip
      %\hskip1.5em plus0em minus0.5em \fi \nobreak
      %\vrule height0.75em width0.5em depth0.25em\fi}  This creates a little box at the end of theorems. etc. However, I already have one defined in one of my packages



\newenvironment{myindentpar}[1]%This indents a whole paragraph: \begin{myindentpar}{3cm}   \end{myindentpar}
{\begin{list}{}
         {\setlength{\leftmargin}{#1}}
         \item[]}
{\end{list}}




\errorcontextlines=0

\renewcommand{\rm}{\normalshape}



\newcommand{\End}{\mathrm{End}}
\newcommand{\sumdd}[2]{\displaystyle \sum_{#1}^{#2}}
\newcommand{\sumd}[1]{\displaystyle \sum_{#1}}
\newcommand{\limd}[1]{\displaystyle \lim_{#1}}
\newcommand{\intd}[1]{\displaystyle \int_{#1}}
\newcommand{\intdd}[2]{\displaystyle \int_{#1}^{#2}}
\newcommand{\fracd}[2]{\displaystyle \frac{#1}{#2}}
\newcommand{\bb}[1]{\mathbb{#1}}
%\newcommand{\cal}[1]{\mathcal{#1}}
\newcommand{\BigFig}[1]{\parbox{10pt}{\Huge #1}}
\newcommand{\BigZero}{\BigFig{0}}
\newcommand{\BigStar}{\BigFig{*}}
\newcommand{\curv}{\mathrm{curv}}
\newcommand{\defeq}{\mathrel{\mathpalette{\vcenter{\hbox{$:$}}}=}}
\newcommand{\SU}{\mathrm{SU}}
\newcommand{\SO}{\mathrm{SO}}
\newcommand{\U}{\mathrm{U}}
\newcommand{\GL}{\mathrm{GL}}
\newcommand{\SL}{\mathrm{SL}}
\newcommand{\PU}{\mathrm{PU}}
\newcommand{\PSL}{\mathrm{PSL}}
\newcommand{\PO}{\mathrm{PO}}
\newcommand{\PSO}{\mathrm{PSO}}
\newcommand{\PGL}{\mathrm{PGL}}
\newcommand{\fl}{{\small \mathrm{flat}}}
\newcommand{\ba}{\mathrm{basic}}
\newcommand{\A}{{\mathcal{A}}}
\newcommand{\G}{{\mathcal{G}}}
\newcommand{\M}{{{M}}}
\newcommand{\Mtwo}{{\mathcal{M}}}
\newcommand{\La}{{{L}}}
\newcommand{\UL}{\underline{\La}}
\newcommand{\e}{\underline{e}}
\newcommand{\rheat}{\mathrm{Heat}}
\newcommand{\NS}{\mathrm{NS}}
\newcommand{\C}{{\mathcal{C}}}
\newcommand{\partialbar}{\overline{\partial}}
\newcommand{\inst}{\mathrm{inst}}
\newcommand{\quilt}{\mathrm{symp}}
\newcommand{\CS}{{\mathcal{CS}}}
\newcommand{\YM}{{\mathcal{YM}}}
\newcommand{\proj}{{\mathrm{proj}}}
\newcommand{\eps}{\epsilon}
\newcommand{\SA}{{\mathcal{SA}}}%for symplectic action
\newcommand{\Ind}{{\mathrm{Ind}}}
\newcommand{\Hom}{{\mathrm{Hom}}}
\newcommand{\D}{{\mathcal{D}}}
\newcommand{\hol}{{\mathrm{hol}}}


%%%%AF Commands
\newcommand{\PSU}{\mathrm{PSU}}
\newcommand{\psu}{\frak{psu}}
\newcommand{\afV}{V}
\newcommand{\afv}{v}
\newcommand{\afr}{r}
\newcommand{\afA}{A}
\newcommand{\afB}{B}
\newcommand{\afa}{a}
\newcommand{\afp}{p}
\newcommand{\afalpha}{\alpha}
\newcommand{\afbeta}{\beta}
\newcommand{\afgamma}{\gamma}
\newcommand{\afphi}{\phi}
\newcommand{\afpsi}{\psi}
\newcommand{\afmu}{\mu}
\newcommand{\afnu}{\nu}
\newcommand{\afU}{U}
\newcommand{\afu}{u}
%%%%%
%%%%%






\setcounter{tocdepth}{2}%Changes whether subsections, etc are displayed in the table of contents.

%\let\oldtocsection=\tocsection  %This and the following lines indent the subsections in the table of contents
%\let\oldtocsubsection=\tocsubsection
%\renewcommand{\tocsection}[2]{\hspace{0em}\oldtocsection{#1}{#2}}
%\renewcommand{\tocsubsection}[2]{\hspace{1em}\oldtocsubsection{#1}{#2}}


%\let\oldtocsection=\tocsection  %This and the following lines indent the subsections in the table of contents
%\let\oldtocsubsection=\tocsubsection
%\let\oldtocsubsubsection=\tocsubsubsection
%\renewcommand{\tocsection}[2]{\hspace{0em}\oldtocsection{#1}{#2}}
%\renewcommand{\tocsubsection}[2]{\hspace{1em}\oldtocsubsection{#1}{#2}}
%\renewcommand{\tocsubsubsection}[2]{\hspace{2em}\oldtocsubsubsection{#1}{#2}}




\title{The linearized holonomy}

\author{David L. Duncan}

\date{}

\begin{document}

\maketitle




\begin{abstract}



\end{abstract}




\tableofcontents



The goal here is to compute the linearized holonomy. Fix a manifold $X$ and a principal $G$-bundle $P \rightarrow X$. Fix preferred points $x \in X$ and $p_x \in P_x$. Fix also a smooth loop

$$\gamma: \left[0, 1 \right] \longrightarrow X$$
with $\gamma(0) = \gamma(1) = x$. 

Given a connection $A$ on $P$, the holonomy around $\gamma$ is
$$\hol_A(\gamma) \in  G,$$
defined as follows. There is a unique $\widetilde{\gamma}: \left[0, 1 \right] \rightarrow \gamma^*P$ covering $\gamma$ satisfying
$$\widetilde{\gamma}(0) = p_x, \indent \iota_{\partial_s \widetilde{\gamma}} A = 0.$$
In the second equation we are viewing $A$ as a $\frak{g}$-valued 1-form on $P$. Then the holonomy is defined to be the unique element of $G$ satisfying
$$p \cdot \hol_A(\gamma) = \widetilde{\gamma}(1).$$


Now fix $V \in  \Omega^1(X, P(\frak{g}))$. We may equally well view $V$ as a basic 1-form on $P$ with values in $\frak{g}$. Let 
$$\hol_A(\gamma)_* V  \defeq \frac{d}{ d \tau} \vert_{\tau = 0} \hol_{A + \tau V} (\gamma) \in \frak{g}$$ 
be the linearization of the holonomy in the direction of $V$.

\begin{proposition}
Suppose $A$ is flat. Then
$$\hol_A(\gamma)_* V = - \intd{\left[0, 1 \right]} \widetilde{\gamma}^* V.$$
\end{proposition}


\begin{proof}
First note that, for each $\tau$, there is a unique $\widetilde{\gamma}_\tau: I \rightarrow \gamma^* P$ covering $\gamma$ and satisfying
\begin{equation}\label{a}
\widetilde{\gamma}_\tau(0) = p_x, \indent (A+ \tau V)_{\widetilde{\gamma}_\tau(s)}(\partial_s \widetilde{\gamma}_\tau) = 0.
\end{equation}
Of course, $\widetilde{\gamma}_0 = \widetilde{\gamma}$. Put that in your back pocket for the time being. 

Now let $\xi$ be the parallel vector field on $P$ generated by $\hol_A(\gamma)_* V $. Then since $A$ is vertical, we have
$$A(\xi)_p = \hol_A(\gamma)_* V $$
for all $p \in P$; in particular, this is the case for $p = \widetilde{\gamma}(1)$. Of course, $A$ is tensorial (it is a 1-form) and so $A(\xi')_p$ makes sense even if $\xi'$ is a vector defined only at $p$. Now $\frac{d}{d\tau} \vert_{\tau = 0} \widetilde{\gamma}(1)$ is such a vector defined at $\widetilde{\gamma}(1)$, and it agrees with $\xi$ at $\widetilde{\gamma}(1)$, by definition. Hence
$$\hol_A(\gamma)_*V = A\left( \frac{d}{d\tau} \vert_{\tau = 0} \widetilde{\gamma}_\tau(1) \right)_{\widetilde{\gamma}(1)}.$$
We have $\frac{d}{d\tau} \vert_{\tau = 0} \widetilde{\gamma}_\tau(0) = 0$ since $\widetilde{\gamma}_\tau(0) = p_x$ for all $\tau$. This gives
$$\begin{array}{rcl}
\hol_A(\gamma)_*V  & = & A\left( \frac{d}{d\tau} \vert_{\tau = 0} \widetilde{\gamma}_\tau(1) \right)_{\widetilde{\gamma}(1)}\\
&& - A\left( \frac{d}{d\tau} \vert_{\tau = 0} \widetilde{\gamma}_\tau(0) \right)_{\widetilde{\gamma}(0)}\\
&&\\
& =& \intdd{0}{1} \partial_s A\left( \partial_\tau \vert_{\tau = 0} \widetilde{\gamma}_\tau\right)_{\widetilde{\gamma}(s)} \: ds\\
&&\\
& = & \intdd{0}{1} A\left( \partial_s  \partial_\tau \vert_{\tau = 0} \widetilde{\gamma}_\tau \right)_{\widetilde{\gamma}(s)} \: ds\\
&& + \intdd{0}{1} dA( \partial_\tau \vert_{\tau = 0} \widetilde{\gamma}_\tau , \partial_s \widetilde{\gamma} )_{\widetilde{\gamma}(s)} \: ds.
\end{array}$$
Now use the definition of curvature $0 = F_A = dA + \frac{1}{2}\left[ A \wedge A \right]$ to get
$$\begin{array}{rcl}
\hol_A(\gamma)_*V  & = & \intdd{0}{1} A\left( \partial_s  \partial_\tau \vert_{\tau = 0} \widetilde{\gamma}_\tau \right)_{\widetilde{\gamma}(s)} \: ds\\
&& - \intdd{0}{1} \left[ A(\partial_\tau \vert_{\tau = 0} \widetilde{\gamma}_\tau )_{\widetilde{\gamma}(s)} , A(\partial_s \widetilde{\gamma} )_{\widetilde{\gamma}(s)} \right] \: ds\\
&&\\
& =&  \intdd{0}{1} A\left( \partial_s  \partial_\tau \vert_{\tau = 0} \widetilde{\gamma}_\tau \right)_{\widetilde{\gamma}(s)} \: ds,
\end{array}$$
where we used (\ref{a}) with $\tau = 0$ in the last line. To compute this, differentiate (\ref{a}) in $\tau$ to get
$$\begin{array}{rcl}
0 & =& A \left( \partial_s \partial_\tau \vert_{\tau = 0} \widetilde{\gamma}_\tau \right)_{\widetilde{\gamma}(s)} + V\left( \partial_s \widetilde{\gamma} \right)_{\widetilde{\gamma}(s)}\\
&& + d A(\partial_s \widetilde{\gamma}, \partial_\tau \vert_{\tau = 0} \widetilde{\gamma} )_{\widetilde{\gamma}(s)}\\
&&\\
& =& A \left( \partial_s \partial_\tau \vert_{\tau = 0} \widetilde{\gamma}_\tau \right)_{\widetilde{\gamma}(s)} + V\left( \partial_s \widetilde{\gamma} \right)_{\widetilde{\gamma}(s)}.
\end{array}$$
Hence
$$\hol_A(\gamma)_*V =  - \intdd{0}{1} V\left( \partial_s \widetilde{\gamma} \right)_{\widetilde{\gamma}(s)} \: ds = - \intd{\left[0, 1\right]} \widetilde{\gamma}^* V,$$
as claimed.
\end{proof}

I want conditions on $V$ that would enable me to say that $V$ is constant along $\widetilde{\gamma}$, and so the integral above equals the value of $V$ at any point on $\widetilde{\gamma}$. This led me to a soul-searching adventure regarding the role closed and exact forms play in evaluating integrals. Here is a summary. Suppose $v$ is a real-valued 1-form on $X$ and $\gamma$ is a path in $X$.

\begin{itemize}
\item The integral $\intd{\gamma} v$ is well-defined regardless of any assumptions. 
\item If $v = df$ is exact, then the above integral equals $f(\gamma(1)) - f(\gamma(0))$. In particular, if $\gamma$ is a loop, then the integral depends only on the image of $v$ in $\Omega^1 / \mathrm{Im} \: d$.
\item If $v$ is closed, then the above integral depends only on the homotopy class of $\gamma$ (rel endpoints, unless $\gamma$ is a loop). 
\item Suppose $\gamma$ is a homologically non-trivial loop, and primitive. Fix any closed form $v_1, \ldots, v_k$ that give a basis for $H^1(X, \bb{Z})$; assume that $v_1$ is Poincar\'{e} dual to $\gamma$, so 
$$\intd{\gamma} v_1 = 1, \indent \intd{\gamma} v_i = 0, \indent i > 1.$$
Now, if $v$ is closed, we can write
$$v = \sumd{i} a_i v_i  + \mu$$
for some real numbers $a_i$, and an exact form $\mu$. This is because the $v_i$ form a basis (over $\bb{R}$) for $H^1(X, \bb{R})$. Then
$$\intd{\gamma} v = a_1.$$
\end{itemize}

This last part is the part I want to generalize to Lie algebra valued forms. Before doing so, note that it has a lot of baggage: the $v_i$'s are irrelevant for $i > 1$. Moreover, all that should matter is the value of $v$ on $\gamma$. So to strip it down to its essentials, this is just saying that $\intd{\gamma} v$ equals a constant times $PD(v_1)\left[ \gamma \right]$, where $v_1$ is a preferred closed 1-form that is Poincar\'{e} dual to $\gamma$. Of course, in this phrasing the whole thing is rather silly. However, I think it becomes less silly when there are Lie algebra coefficients involved. 

That is, I want to leave $v_1$ as a preferred closed 1-form (real valued) that is Poincar\'{e} dual to $\gamma$, but I want $V$ to be an element of $\Omega^1(X, P(\frak{g}))$ satisfying 
$$d_A V = 0.$$
Assume $A$ is flat. By adding an exact form, we may assume $v_1$ satisfies
$$v_1 \vert_{\gamma} = d s,$$
where $s$ is the parameter for the domain of $\gamma$. That is, view $\gamma$ as a map
$$S^1 \longrightarrow X.$$
Then we may assume the pullback $\gamma^* v_1$ equals $d s$ (let's say the circle has length 1 relative to this 1-form). Next, since $A$ is flat, we can trivialize $P \vert_{\gamma}$ in such a way that
$$d_A\vert_{\gamma} = d + H d s.$$
Here $H \in \frak{g}$ is such that $\exp(-H)$ is the holonomy around $\gamma$; alternatively, $A\vert_\gamma$ is the 1-form on $\gamma \times G$ sending vectors of the form $(\mu, g \xi ) \in T\gamma \times g \frak{g}$ to
$$H \iota_\mu d s + \xi.$$
(In this latter formulation, it is easier to see that $H$ is the holonomy. Indeed, the loop
$$\widetilde{\gamma}(s) = (\gamma(s),  \exp(-sH))$$
is such that its derivative satisfies
$$A( \partial_s \widetilde{\gamma} ) = H  - H= 0,$$
and so is the horizontal lift of $\gamma$. Then $\widetilde{\gamma}(1) = \exp(-H)$ is the holonomy.)

Next, we have 
$$0 = d_A V \vert_\gamma = d s \wedge \partial_s V + d s \wedge \left[ H , V \right] .$$
Hence
$$\partial_s V = -\left[H, V \right].$$
Since $H$ is constant, we can integrate this to get
$$V_ = \mathrm{Ad}(\exp(-s H)) V_0.$$
Then
$$\begin{array}{rcl}
\hol_A(\gamma)_* V & = & - \intd{\left[ 0, 1 \right]} \widetilde{\gamma}^* V \\
&&\\
& = & - \intd{\left[ 0, 1 \right]}  V_{s} (\partial_s \gamma) \; ds \\
&&\\
& = & - \intd{\left[ 0, 1 \right]} \mathrm{Ad}(\exp(-s H)) V_0(\partial_s \gamma) \; d s
\end{array}$$
In the last line we used the fact that $V$ is horizontal, and so only reads off the $\partial_s \gamma$ part of $\partial_s \widetilde{\gamma}$. Of course, since we are parametrizing with $\gamma$, the vector $\partial_s \gamma$ is constant in $s$. This gives
$$\hol_A(\gamma)_* V = - \intd{\left[ 0, 1 \right]} \mathrm{Ad}(\exp(-s H)) \; d s  \; V_0(\partial_s \gamma).$$ 
This integral can be viewed as an integral of a path of linear operators on $\frak{g}$. The point of all of this is that we have recovered a formula that expresses the linearized holonomy entirely in terms of the holonomy $H$ of $A$, and the value of $V$ at the base point of $\gamma$, in the direction of $\partial_s \gamma$. In particular, if the holonomy of $A$ around $\gamma$ were trivial, then this shows
$$\hol_A(\gamma)_* V = \iota_{\partial_s \gamma(0)} V_{\gamma(0)}  \in P(\frak{g})_x \cong \frak{g},$$
with the preferred point $p \in P$ lying over $\gamma(0)$ being used to identify the fiber $P(\frak{g})_x$ with $\frak{g}$. (This assumes that $V$ is covariantly constant along $\gamma$.)
\end{document}

